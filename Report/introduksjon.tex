\section{Introduction}
- Some information about the purpose of the research\\
- Some information about Inter Rater Reliability and Cohen's Kappa test\\
- Discuss some information about reliability, cost,\\[30pt]

An introduction is important to ensure the background of the report is sufficiently explored, and to give the readers up-to-date information before jumping into the content of the report. The main parts of an introduction should be a background of the content that will be covered in the report, the motivation of the project or purpose of the report, and finally the outline of the report. The outline of the report is important to let readers know what to expect by reading the report, what sections will be covered and a brief outlook on what it entails.

The page margins can be set under the package \textit{geometry}. It is advisable to look up the margin requirements of your department. 

If you are not familiar with concepts like styles, captioning, cross-referencing, and how to generate tables of contents, figures etc. in LaTeX, the Overleaf guides are a useful start at: \url{https://www.overleaf.com/learn/latex/Learn_LaTeX_in_30_minutes}. 

Otherwise, a lot of help can be obtained through Stack Exchange, where it is a forum that people get help from others regarding \LaTeX\ writing. 

Other useful tools are
\begin{itemize}
    \item Detexify, which can convert your drawings to \LaTeX\ commands, 
    \item Mathpix tools, which you can use to screenshot an equation and it will convert into \LaTeX\ equation form to be copied into your .tex document,
    \item Grammarly is a great help for grammar assistance on overleaf. 
\end{itemize}

\subsection{Sectioning}
This is a subsection. You can also use sub-subsections, although it isn't recommended.

\subsubsection{Subsubsection}
Example of a sub-subsection. It isn't recommended to nest one more layer. However, if needed, the command is \textbackslash paragraph\{\textit{insert section name}\}. It should not be numbered. 

\vspace{1em} 
Finally, make sure to outline the report at the end of the introduction section. 
