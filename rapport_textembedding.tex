\documentclass[11pt,a4paper,oldfontcommands]{memoir}
\usepackage[utf8]{inputenc}
\usepackage[T1]{fontenc}
\usepackage{microtype}
\usepackage[dvips]{graphicx}
%\usepackage{xcolor}
\usepackage{times}
\usepackage{amsmath}
\usepackage{multirow}
%\usepackage{amsfonts}
%\usepackage{amssymb}

\usepackage[dvipsnames]{xcolor}
%\usepackage{tikz}
%\usetikzlibrary{ shapes.geometric }
%\usetikzlibrary{calc}
%\usepackage{anyfontsize}
\usepackage{mathtools,amsfonts,amssymb,amsthm}
\usepackage{graphicx}

\usepackage{pdfpages}

\usepackage{wrapfig}

\usepackage[
breaklinks=true,colorlinks=true,
%linkcolor=blue,urlcolor=blue,citecolor=blue,% PDF VIEW
linkcolor=black,urlcolor=black,citecolor=black,% PRINT
bookmarks=true,bookmarksopenlevel=2]{hyperref}

\usepackage{geometry}
% PDF VIEW
% \geometry{total={210mm,297mm},
% left=25mm,right=25mm,%
% bindingoffset=0mm, top=25mm,bottom=25mm}
% PRINT
\geometry{
left=20mm,right=20mm, top = 25mm, bottom = 25mm}%,
%bindingoffset=10mm, top=25mm,bottom=25mm}

\OnehalfSpacing
%\linespread{1.3}

%%% CHAPTER'S STYLE
\chapterstyle{bianchi}
%\chapterstyle{ger}
%\chapterstyle{madsen}
%\chapterstyle{ell}
%%% STYLE OF SECTIONS, SUBSECTIONS, AND SUBSUBSECTIONS
\renewcommand*{\chapnamefont}{\normalfont\large\rmfamily\itshape}
%\renewcommand*{\chapnumfont}{\normalfont\large}
%\renewcommand*{\chaptitlefont}{\normalfont\huge\rmfamily}

%%% STYLE OF PAGES NUMBERING
%\pagestyle{companion}\nouppercaseheads 
%\pagestyle{headings}
%\pagestyle{Ruled}
\pagestyle{plain}
\makepagestyle{plain}
\makeevenfoot{plain}{}{}{\thepage}
\makeoddfoot{plain}{}{}{\thepage}
\makeevenhead{plain}{}{}{}
\makeoddhead{plain}{}{}{}

\maxtocdepth{subsection} % chapters, sections, and subsections are in the Table of Contents

\newtheorem*{theorem}{Teorem}
\newtheorem*{lemma}{Lemma}

\renewcommand{\qedsymbol}{$\square$}
\renewcommand*{\proofname}{Bevis}

\newcommand{\R}{{\mathbb R}}
\newcommand{\N}{{\mathbb N}}
\newcommand{\Z}{\mathbb Z}
\newcommand{\Q}{\mathbb Q}
\newcommand{\ra}{\rightarrow}


\begin{document}

%%%%forside
\begin{center}
    \vspace{2cm}
    \Huge{\textbf{Can we demonstrate the centroid method is better than using an LLM directly to do similar coding tasks? }}\\
    \vspace{0.6cm}
    \LARGE{Navn}\\
    \vspace{0.25cm}
    \LARGE{\textit{Subtitle}}\\
\end{center}

\section*{Introduction}

\section*{Method}
\subsection*{First "project"}
\begin{flushleft}
We started by creating our own dataset centered around the question "Why did you choose the Honours program".
We then created 25 fictitious responses to this question, which made up our dataset. We then collectively decided on categories 
(should we write out the categories?) and sorted each response onto one of these categories to form our "ground truth" that we later 
compared our other results to. \\
Based on this dataset, we prompted ChatGPT UiO in different ways. \\
- We started by asking it to simply sort the responses into categories - (made up a category for each - this should be in results)\\
- we then proceeded to repeat the same process, but gave it additional instructions.\\[10pt]

Then we gave it the following prompt: (prompt) and repeated this process three times. The data was collected, sorted into a list and each category was assigned
a number for ease of computation. We then proceeded to calculate the Cohen Kappa Score using the ($cohen_kappa_score$)-function from (scikitlearn.metrics) using the following code:\\
(code)

We then used the centroid method on this dataset using the code provided from the Colab. 

\subsection*{Second "project"}
For this project we used a pre-categorized dataset from Hugging Face (link), containing news headlines which had been labeled with the following categories: "politics", "business", "health", "sports", "tech" and "entertainment".
The dataset had already been split into a training- and test-dataset, and we used the test-dataset which consisted of 828 headlines and used the first 500. \\
We then proceeded to manually prompt ChatGPT UiO by asking it to categorize the first 500 headlines by giving it chunks of 50 at a time and starting a new chat per chunk. This prompting 
was done using the following prompt:\\
(prompt)\\
The prompt had to be somewhat modified in the process, since the format of the output from GPT varied, despite the specifications in the prompt. (Explain further)\\[10pt]
The dataset of the 500 first datapoints were coded by ChatGPT UiO three separate times and each run was collected into a single document. This text was processed using the following code:\\
(code)\\
The produced lists were then used as a basis for calculating the Inter Rater Reliability using Cohen's Kappa test using the same procedure as we did in (first project).

We also used the centroid method to obtain a categorization. The centroid method gives a distribution 


\section*{Results}

\section*{Discussion}

\section*{Conclusion}
    
\end{flushleft}

\end{document}