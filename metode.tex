\section{Methodology}
Elaborate in detail the methods you used to collect the data. This should also include any potential uncertainty and errors. 
Any relevant equations should be explained either here or in the background section. 

Typically, scientific writing should be in the passive voice. The use of active voice has been more and more accepted in scientific writing, however, it is generally better to be in a passive voice. 
A passive voice removes the author from the work, which makes the work done feel more generalised, and should be replicable by anybody. 
Since scientific work should be factual by nature, the results should be the same regardless of the individual doing the experiment. 
Although this is not always true, especially considering bias, it is generally recommended to leave the subject out of the report. 

\subsection*{First "project"}
\begin{flushleft}
We started by creating our own dataset centered around the question "Why did you choose the Honours program".
We then created 25 fictitious responses to this question, which made up our dataset. We then collectively decided on categories 
(should we write out the categories?) and sorted each response onto one of these categories to form our "ground truth" that we later 
compared our other results to. \\
Based on this dataset, we prompted ChatGPT UiO in different ways. \\
- We started by asking it to simply sort the responses into categories - (made up a category for each - this should be in results)\\
- we then proceeded to repeat the same process, but gave it additional instructions.\\[10pt]

Then we gave it the following prompt: (prompt) and repeated this process three times. The data was collected, sorted into a list and each category was assigned
a number for ease of computation. We then proceeded to calculate the Cohen Kappa Score using the ($cohen_kappa_score$)-function from (scikitlearn.metrics) using the following code:\\
(code)

We then used the centroid method on this dataset using the code provided from the Colab. - have to specify the parameters used

\subsection*{Second "project"}
For this project we used a pre-categorized dataset from Hugging Face (link), containing news headlines which had been labeled with the following categories: "politics", "business", "health", "sports", "tech" and "entertainment".
The dataset had already been split into a training- and test-dataset, and we used the test-dataset which consisted of 828 headlines and used the first 500. \\
We then proceeded to manually prompt ChatGPT UiO by asking it to categorize the first 500 headlines by giving it chunks of 50 at a time and starting a new chat per chunk. This prompting 
was done using the following prompt:\\
(prompt)\\
The prompt had to be somewhat modified in the process, since the format of the output from GPT varied, despite the specifications in the prompt. (Explain further)\\[10pt]
The dataset of the 500 first datapoints were coded by ChatGPT UiO three separate times and each run was collected into a single document. This text was processed using the following code:\\
(code)\\
The produced lists were then used as a basis for calculating the Inter Rater Reliability using Cohen's Kappa test using the same procedure as we did in (first project).

We also used the centroid method to obtain a categorization. The centroid method gives its result as a distribution between the different categories. We wrote a script to obtain only the most probable category two different ways.
First we set $\alpha = 150$ to ~shift the weight to get a value greater than 0.50 for all of them~ and compared this categorization to the ground truth using Cohen's Kappa Score using the following code:\\
(code)\\
We then set $\alpha = 15$ and excluded the data where the greatest value in the distribution was smaller than 0.50, and constructed a ground truth where we excluded the same values, 
and used Cohen's Kappa Score to compare the two using the following code:\\
(code)\\

Llama\\
Categorization was done using Llama3 but using prompting in the terminal using a Powershell script, and using 3"cloud server".\\
The prompting in the Powershell was done using the following code:\\
(code)
The prompting with the 3"cloud server" was done using the following code:\\
(code)

\end{flushleft}

\subsection*{Third "project"}